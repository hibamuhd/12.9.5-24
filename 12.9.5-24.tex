\documentclass[12pt]{article}
\usepackage{amsmath}

\begin{document}

\title{Discreet 12.9.5.24}
\author{Hiba Muhammed \\
        EE23BTECH11026}
\maketitle

\section*{Problem Statement}
If \(S_1\), \(S_2\), \(S_3\) are the sum of the first \(n\) natural numbers, their squares, and their cubes, respectively, show that 

\[ 9(S\scriptstyle 2)^2 = (S\scriptstyle 3)(1 + 8(S\scriptstyle 1)) \]

\section*{Solution}

\begin{table}[h]
  \centering
  \begin{tabular}{|c|c|c|}
    \hline
    \textbf{Sequence} & \textbf{Expression} & \textbf{Description} \\
    \hline
    \(s_1\) & \(\frac{n(n+1)}{2}\) & sum of n natural numbers\\
    \hline
    \(s_2\) & \(\frac{n(n+1)(2n+1)}{6}\) & sum of squares\\
    \hline
    \(s_3\) & \(\left(\frac{n(n+1)}{2}\right)^2\) & sum of cubes \\
    \hline
    \(x_1\) & \(x_1\brak{n} = n u\brak{n}\) & \\
    \hline
    \(x_2\) & \(x_2\brak{n} = n^2 u\brak{n}\) &  \\
    \hline
    \(x_3\) & \(x_3\brak{n} = n^3 u\brak{n}\) & \\
    \hline
\end{tabular}


  \caption{Input Equations}
  \label{tab:input-equations}
  
\end{table}

Now, let's substitute these expressions into the given equation \(9(S_2)^2 = (S_3)(1 + 8(S_1))\) and simplify:

\begin{align*}
    9\left(\frac{n(n+1)(2n+1)}{6}\right)^2 &= \left(\frac{n(n+1)}{2}\right)^2 \left(1 + 8 \cdot \frac{n(n+1)}{2}\right) \\
    \frac{9}{36}\left(n(n+1)(2n+1)\right)^2 &= \frac{1}{4}\left(n(n+1)\right)^2 \left(1 + 4n(n+1)\right) \\
    \frac{1}{4}\left(n(n+1)(2n+1)\right)^2 &= \frac{1}{4}\left(n(n+1)\right)^2 \left(4n(n+1) + 1\right) \\
    \left(n(n+1)(2n+1)\right)^2 &= \left(n(n+1)\right)^2 \left(4n(n+1) + 1\right) \\
\end{align*}

The last equation holds true, which verifies that \(9(S_2)^2 = (S_3)(1 + 8(S_1))\) for the given expressions of \(S_1\), \(S_2\), and \(S_3\).

\end{document}

