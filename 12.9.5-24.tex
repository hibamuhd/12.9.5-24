\let\negmedspace\undefined
\let\negthickspace\undefined
\documentclass[journal,12pt,onecolumn]{IEEEtran}
\usepackage{cite}
\usepackage{amsmath,amssymb,amsfonts,amsthm}
\usepackage{algorithmic}
\usepackage{graphicx}
\usepackage{textcomp}
\usepackage{xcolor}
\usepackage{txfonts}
\usepackage{listings}
\usepackage{enumitem}
\usepackage{mathtools}
\usepackage{gensymb}

\usepackage{tkz-euclide} % loads  TikZ and tkz-base
\usepackage{listings}



\newtheorem{theorem}{Theorem}[section]
\newtheorem{problem}{Problem}
\newtheorem{proposition}{Proposition}[section]
\newtheorem{lemma}{Lemma}[section]
\newtheorem{corollary}[theorem]{Corollary}
\newtheorem{example}{Example}[section]
\newtheorem{definition}[problem]{Definition}
%\newtheorem{thm}{Theorem}[section] 
%\newtheorem{defn}[thm]{Definition}
%\newtheorem{algorithm}{Algorithm}[section]
%\newtheorem{cor}{Corollary}
\newcommand{\BEQA}{\begin{eqnarray}}
\newcommand{\EEQA}{\end{eqnarray}}
\newcommand{\system}[1]{\stackrel{#1}{\rightarrow}}

\newcommand{\define}{\stackrel{\triangle}{=}}
\theoremstyle{remark}
\newtheorem{rem}{Remark}
%\bibliographystyle{ieeetr}
\begin{document}
%
\providecommand{\pr}[1]{\ensuremath{\Pr\left(#1\right)}}
\providecommand{\prt}[2]{\ensuremath{p_{#1}^{\left(#2\right)} }}        % own macro for this question
\providecommand{\qfunc}[1]{\ensuremath{Q\left(#1\right)}}
\providecommand{\sbrak}[1]{\ensuremath{{}\left[#1\right]}}
\newcommand{\brac}[1]{\left( #1 \right)}
\providecommand{\lsbrak}[1]{\ensuremath{{}\left[#1\right.}}
\providecommand{\rsbrak}[1]{\ensuremath{{}\left.#1\right]}}
\providecommand{\brak}[1]{\ensuremath{\left(#1\right)}}
\providecommand{\lbrak}[1]{\ensuremath{\left(#1\right.}}
\providecommand{\rbrak}[1]{\ensuremath{\left.#1\right)}}
\providecommand{\cbrak}[1]{\ensuremath{\left\{#1\right\}}}
\providecommand{\lcbrak}[1]{\ensuremath{\left\{#1\right.}}
\providecommand{\rcbrak}[1]{\ensuremath{\left.#1\right\}}}
\newcommand{\sgn}{\mathop{\mathrm{sgn}}}
\providecommand{\abs}[1]{\left\vert#1\right\vert}
\providecommand{\res}[1]{\Res\displaylimits_{#1}} 
\providecommand{\norm}[1]{\left\lVert#1\right\rVert}
%\providecommand{\norm}[1]{\lVert#1\rVert}
\providecommand{\mtx}[1]{\mathbf{#1}}
\providecommand{\mean}[1]{E\left[ #1 \right]}
\providecommand{\cond}[2]{#1\middle|#2}
\providecommand{\fourier}{\overset{\mathcal{F}}{ \rightleftharpoons}}
\newenvironment{amatrix}[1]{%
  \left(\begin{array}{@{}*{#1}{c}|c@{}}
}{%
  \end{array}\right)
}
%\providecommand{\hilbert}{\overset{\mathcal{H}}{ \rightleftharpoons}}
%\providecommand{\system}{\overset{\mathcal{H}}{ \longleftrightarrow}}
    %\newcommand{\solution}[2]{\textbf{Solution:}{#1}}
\newcommand{\solution}{\noindent \textbf{Solution: }}
\newcommand{\cosec}{\,\text{cosec}\,}
\providecommand{\dec}[2]{\ensuremath{\overset{#1}{\underset{#2}{\gtrless}}}}
\newcommand{\myvec}[1]{\ensuremath{\begin{pmatrix}#1\end{pmatrix}}}
\newcommand{\mydet}[1]{\ensuremath{\begin{vmatrix}#1\end{vmatrix}}}
\newcommand{\myaugvec}[2]{\ensuremath{\begin{amatrix}{#1}#2\end{amatrix}}}
\providecommand{\rank}{\text{rank}}
\providecommand{\pr}[1]{\ensuremath{\Pr\left(#1\right)}}
\providecommand{\qfunc}[1]{\ensuremath{Q\left(#1\right)}}
    \newcommand*{\permcomb}[4][0mu]{{{}^{#3}\mkern#1#2_{#4}}}
\newcommand*{\perm}[1][-3mu]{\permcomb[#1]{P}}
\newcommand*{\comb}[1][-1mu]{\permcomb[#1]{C}}
\providecommand{\qfunc}[1]{\ensuremath{Q\left(#1\right)}}
\providecommand{\gauss}[2]{\mathcal{N}\ensuremath{\left(#1,#2\right)}}
\providecommand{\diff}[2]{\ensuremath{\frac{d{#1}}{d{#2}}}}
\providecommand{\myceil}[1]{\left \lceil #1 \right \rceil }
\newcommand\figref{Fig.~\ref}
\newcommand\tabref{Table~\ref}
\newcommand{\sinc}{\,\text{sinc}\,}
\newcommand{\rect}{\,\text{rect}\,}
%%
%   %\newcommand{\solution}[2]{\textbf{Solution:}{#1}}
%\newcommand{\solution}{\noindent \textbf{Solution: }}
%\newcommand{\cosec}{\,\text{cosec}\,}
%\numberwithin{equation}{section}
%\numberwithin{equation}{subsection}
%\numberwithin{problem}{section}
%\numberwithin{definition}{section}
%\makeatletter
%\@addtoreset{figure}{problem}
%\makeatother

%\let\StandardTheFigure\thefigure
\let\vec\mathbf


\bibliographystyle{IEEEtran}
\title{Discreet 12.9.5.24}
\author{HIBA MUHAMMED\\
        EE23BTECH11026}
\maketitle

\section*{Problem Statement}
If \(S_1\), \(S_2\), \(S_3\) are the sum of the first \(n\) natural numbers, their squares, and their cubes, respectively, show that 
\[ 9(S\scriptstyle 2)^2 = (S\scriptstyle 3)(1 + 8(S\scriptstyle 1)) \]

\section*{Solution}
\begin{table}[h]
  \centering
  \begin{tabular}{|c|c|c|}
    \hline
    \textbf{Sequence} & \textbf{Expression} & \textbf{Description} \\
    \hline
    \(s_1\) & \(\frac{n(n+1)}{2}\) & sum of n natural numbers\\
    \hline
    \(s_2\) & \(\frac{n(n+1)(2n+1)}{6}\) & sum of squares\\
    \hline
    \(s_3\) & \(\left(\frac{n(n+1)}{2}\right)^2\) & sum of cubes \\
    \hline
    \(x_1\) & \(x_1\brak{n} = n u\brak{n}\) & \\
    \hline
    \(x_2\) & \(x_2\brak{n} = n^2 u\brak{n}\) &  \\
    \hline
    \(x_3\) & \(x_3\brak{n} = n^3 u\brak{n}\) & \\
    \hline
\end{tabular}


  \caption{Input Equations}
  \label{tab:input-equations}
  
\end{table}
 By the differentiation property :
    \begin{align}
     n x\brak{n} & \system{Z} \brak{-z} \frac{dX\brak{z}}{dz} \label{eq:11.9.5.26.1}\\
    \implies    n u\brak{n} & \system{Z} \frac{z^{-1}}{\brak{1-z^{-1}}^2} ,   \abs{z} >1 \label{eq:11.9.5.26.2}\\
    \implies     n^2 u\brak{n} & \system{Z} \frac{z^{-1}\brak{z^{-1}+1}}{\brak{1-z^{-1}}^3} ,  \abs{z} > 1\label{eq:11.9.5.26.3}\\
    \implies     n^3 u\brak{n} & \system{Z} \frac{z^{-1}\brak{1+4z^{-1}+z^{-2}}}{\brak{1-z^{-1}}^4} ,   \abs{z} >1\label{eq:11.9.5.26.4} 
    \end{align}
    
\begin{align}
    X_1(z) &= \frac{z^{-1}}{(1-z^{-1})^2}, &|z| > 1 \\
    X_2(z) &= \frac{z^{-1}(z^{-1}+1)}{(1-z^{-1})^3}, &|z| > 1 \\
    X_3(z) &= \frac{z^{-1}(1+4z^{-1}+z^{-2})}{(1-z^{-1})^4}, &|z| > 1 
\end{align} 
    The convolution sum is defined as
    \begin{align}
        y(n) &= x(n)* h(n) = \sum_{k=-\infty}^{\infty}x\brak{k}h\brak{n-k}\\
        x(n)*u(n)  &= \sum_{k=0}^{n}x\brak{k}\\
        y(n) &= x(n) * u(n) 
    \end{align}
\begin{align}
    Y(z) &= X(z) \cdot u(z) \\
    Y_1(z) &= \frac{z^{-1}}{(1-z^{-1})^3} \\
    Y_2(z) &= \frac{z^{-1}(z^{-1}+1)}{(1-z^{-1})^4} \\
    Y_3(z) &= \frac{z^{-1}(1+4z^{-1}+z^{-2})}{(1-z^{-1})^5}
\end{align} 
    

\( y(n) \) from the inverse Z-transforms of \( Y_1(z) \), \( Y_2(z) \), and \( Y_3(z) \).
\begin{align}
& y_1(n) = \delta(n-2)\\ 
&  y_2(n) = -u(n-1) + 3u(n-2) - 6u(n-3) + 4u(n-4)\\ 
& y_3(n) = -\frac{1}{4}u(n-1) -\frac{3}{8}u(n-2) -\frac{1}{4}u(n-3) + \delta(n-4) \\
& 9(y_2)^2 = (y_3)(1 + 8(y_1)) \\
& 9(-u(n-1) + 3u(n-2) - 6u(n-3) + 4u(n-4))^2 = (-\frac{1}{4}u(n-1) -\frac{3}{8}u(n-2) -\frac{1}{4}u(n-3) + \delta(n-4))(1 + 8(\delta(n-2))) \\
& 9(-u(n-1) + 3u(n-2) - 6u(n-3) + 4u(n-4))^2 = 9u(n-1)^2 - 54u(n-1)u(n-2) + 108u(n-1)u(n-3) - 72u(n-1)u(n-4) + 81u(n-2)^2 - 324u(n-2)u(n-3) + 216u(n-2)u(n-4) + 144u(n-3)^2 - 576u(n-3)u(n-4) + 256u(n-4)^2 \\
& (-\frac{1}{4}u(n-1) -\frac{3}{8}u(n-2) -\frac{1}{4}u(n-3) + \delta(n-4))(1 + 8(\delta(n-2))) = -\frac{1}{4}u(n-1) -\frac{3}{8}u(n-2) -\frac{1}{4}u(n-3) + \delta(n-4) -2u(n-1)\delta(n-2) -6u(n-2)\delta(n-2) -2u(n-3)\delta(n-2) + 8\delta(n-4)\delta(n-2)
\end{align}
the coefficients of the terms on both sides are the same, which means that the identity holds.
\[ 9(y_2)^2 = (y_3)(1 + 8(y_1)) \]
    
\end{document}

